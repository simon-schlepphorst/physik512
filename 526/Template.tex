\documentclass[11pt, ngerman, fleqn, DIV=15, headinclude, BCOR=2cm]{scrreprt}

\usepackage{../../header}

\usepackage{placeins}
%\usepackage[maxfloats=50]{morefloats}

\usepackage{csquotes}

\usepackage{tikz}
\usetikzlibrary{chains}
\usetikzlibrary{shapes.geometric}

\tikzset{device/.style={
                rectangle,
                minimum size=6mm,
                draw=black
            },
            monitor/.style={
                rectangle,
                rounded corners=2mm,
                minimum size=6mm,
                draw=black
            },
        }

\usepackage{pgfplots}
\pgfplotsset{
    compat=1.9,
    width=0.8\linewidth,
    xticklabel style={/pgf/number format/use comma},
    yticklabel style={/pgf/number format/use comma},
}
\usepgfplotslibrary{polar}

\usepgfplotslibrary{external}
\tikzexternalize[mode=list and make]
\tikzsetexternalprefix{Abbildung-}

\DeclareSIUnit{\skt}{SKT}

\usepackage{booktabs}

\hypersetup{
    pdftitle=
}

\newcommand{\plotwidth}{0.8\linewidth}

\subject{Praktikumsprotokoll}
\title{Compton-Effekt}
\subtitle{Versuch P526 -- Universität Bonn}
\author{
	Frederike Schrödel \\
	\small{\href{mailto:fschroedel@gmx.de}{fschroedel@gmx.de}}
	\and
	Simon Schlepphorst \\
	\small{\href{mailto:s2@uni-bonn.de}{s2@uni-bonn.de}}
}

\date{2015-11-10}

\publishers{Tutor: Peter Klassen
}

\begin{document}

\maketitle

\begin{abstract}
%TODO
\end{abstract}


\tableofcontents

\chapter{Theorie}

%TODO

\chapter{Durchführung}

\section{Einstellen des Verstärkers}
\begin{figure}[h]
    \centering
    \includegraphics[width=\plotwidth]{plot_fit_peak_ohne}
    \caption{%
	    $^{137}\text{Cs}$-Spektrum ohne Aluminium Absorber
   }
    \label{fig:plot_fit_peak_ohne}
\end{figure}

%TODO

\chapter{Auswertung}

\section{Totaler Wirkungsquerschnitt}

Es werden die Messdaten aus \fehlt%TODO\ref{}
benutzt.
An die Photopeaks der Spektren passen wir eine Gaussfunktion an. Das Resultat
ist vergrößert in Abbildung~\ref{fig:amplituden} zu sehen. Die gesamten
Spektren kann man in Abbildung~\ref{fig:plot_fit_peak_ohne} und im
Anhang~\ref{anhang-wirkungsquerschnitt} betrachten.

\begin{equation}
	g\del x = \frac{a}{\varsigma \sqrt{2 \pi}} \exp\del{\frac{\del
		{x - x_0}^2}{2 \varsigma^2}}
		\quad \quad \quad \text{mit FWHM} \approx 2.355 \varsigma
\end{equation}

Die Flächen unter den Gaußkurven sind auf die Kurve ohne Absorber normiert als
Intensitäten in Tabelle~\ref{tab:amplituden} aufgeführt. An diese wurde in
Abbildung~\ref{fig:cross-section} eine Exponentialfunktion der folgenden Form
angepasst:

\begin{equation}
	N\del x = N_0 \exp\del{-\sigma \rho_T x} \quad \quad \quad \text{mit }
	\rho_T = \frac{N_A \rho}{M}
\end{equation}

Mit dem Ergenis aus der Anpassung und den Literaturwerten für Aluminium ($\rho
= \SI{<< alu_dichte >>}{\gram\per\cubic\centi\metre}$, $M = \SI{<< alu_amu >>}
{\atomicmassunit}$) ergibt sich für den totalen Wirkungsquerschnitt $\sigma$:

\begin{align*}
	&\sigma \rho_T = \SI{<< alu_a >>}{\per\milli\metre}
	&&\rho_T = \SI{<< alu_n >>}{\per\cubic\milli\metre}
	&&\implies \sigma = \SI{<< alu_sigma >>}{\barn}
\end{align*}

\fehlt %TODO Diskussion

\begin{figure}
    \centering
    \includegraphics[width=\plotwidth]{total-cross-section-data}
    \caption{%
	    Photopeaks im Spektrum der eingebauten $^{137}\text{Cs}$-Quelle ohne
	    und mit \SIlist{1;5;10;20}{\milli\meter} Aluminium Absorber
    }
    \label{fig:amplituden}
\end{figure}

\begin{table}
    \centering
    \begin{tabular}{SSSS}
        {Dicke / \si{\milli\meter}} &
        {Scheitelpunkt / Kanal} &
        {FWHM / Kanal} &
	{rel. Intensität} \\
        \midrule
        %< for row in total_cross_section_table: ->%
        << ' & '.join(row) >> \\
        %< endfor ->%
    \end{tabular}
    \caption{%
        Anpassungsparameter für die verschiedenen Dicken der
        Absorbermaterialien.
    }
    \label{tab:amplituden}
\end{table}

\begin{figure}
    \centering
    \includegraphics[width=\plotwidth]{total-cross-section-fit}
    \caption{%
	    Transmittierte relative Intensität abhängig von der Absoberdicke
	    mit angepasster Exponentialfunktion
    }
    \label{fig:cross-section}
\end{figure}

\clearpage

\section{Energiekalibrierung}

Mit den Messdaten aus \fehlt%TODO\ref{ }
und der Messung ohne Absober aus \fehlt%TODO\ref{ }
wird nun die Energiekalibrierung durchgeführt. Dazu werden an die Spitzen der
Spektren Gaussfunktionen angepasst. Das Ergebnis ist in
Abbildung~\ref{fig:plot_fit_peak_ohne} und Abbildungen~\fehlt%TODO\ref{ }
zu sehen. Außerdem sind die Parameter der Gaußfunktionen nocheinmal in
Tabelle~\ref{tab:energiekalibrierung}
aufgeführt.

\begin{figure}
	\centering
	\begin{tabular}{SSS}
		{Scheitelpunkte / Kanal} &
		{FWHM / Kanal} &
		{Energie / \si{\kilo\electronvolt}}\\
		\midrule
		%< for row in energy_calibration_table: ->%
		<< ' & '.join(row) >> \\
		%< endfor ->%
	\end{tabular}
	\caption{%
		Anpassungeparameter für die Energiekalibrierung
	}
	\label{tab:energiekalibrierung}
\end{figure}


%TODO

\chapter{Ergebnis}

%TODO


%%%%%%%%%%%%%%%%%%%%%%%%%%%%%%%%%%%%%%%%%%%%%%%%%%%%%%%%%%%%%%%%%%%%%%%%%%%%%%%
%                                   Anhang                                    %
%%%%%%%%%%%%%%%%%%%%%%%%%%%%%%%%%%%%%%%%%%%%%%%%%%%%%%%%%%%%%%%%%%%%%%%%%%%%%%%

\begin{appendix}

\chapter{Anhang}

\section{Abbildungen zur Bestimmung des totalen Wirkungsquerschnitts} \label{anhang-wirkungsquerschnitt}
\begin{figure}[h]
    \centering
    \includegraphics[width=0.6\textwidth]{plot_fit_peak_01mm}
    \caption{%
	    $^{137}\text{Cs}$-Spektrum mit \SI{1}{\milli\metre} Aluminium
	    Absorber
    }
    \label{fig:plot_fit_peak_01mm}
\end{figure}

\begin{figure}[h]
    \centering
    \includegraphics[width=0.6\textwidth]{plot_fit_peak_05mm}
    \caption{%
	    $^{137}\text{Cs}$-Spektrum mit \SI{5}{\milli\metre} Aluminium
	    Absorber
   }
    \label{fig:plot_fit_peak_05mm}
\end{figure}

\begin{figure}[h]
    \centering
    \includegraphics[width=0.6\textwidth]{plot_fit_peak_10mm}
    \caption{%
	    $^{137}\text{Cs}$-Spektrum mit \SI{10}{\milli\metre} Aluminium
	    Absorber
   }
    \label{fig:plot_fit_peak_10mm}
\end{figure}

\begin{figure}[h]
    \centering
    \includegraphics[width=0.6\textwidth]{plot_fit_peak_20mm}
    \caption{%
	    $^{137}\text{Cs}$-Spektrum mit \SI{20}{\milli\metre} Aluminium
	    Absorber
   }
    \label{fig:plot_fit_peak_20mm}
\end{figure}

%TODO

\end{appendix}

\end{document}

% vim: spell spelllang=de tw=79
